%-------------------------------------------------------------------------------
%	SECTION TITLE
%-------------------------------------------------------------------------------
\cvsection{Projects}

% \href{https://github.com/\@github}{\faGithubSquare\acvHeaderIconSep\@github}%
%-------------------------------------------------------------------------------
%	CONTENT
%-------------------------------------------------------------------------------
\begin{cventries}

%---------------------------------------------------------
  \cventry
    {\textbf{Flask, Javascript, MySQL}} % Job title
    {Tekken 7 Data Index} % Organization
    {\href{https://github.com/basulaib/TekkenProject}{\faGithub /TekkenProject}} % Location
    {} % Date(s)
    {
      \begin{cvitems} % Description(s) of tasks/responsibilities
        %\item {Stack: Flask, Javascript, MySQL}
        \item {Present data (input, frames, etc) in a friendly easy-to-read form, even to unfamiliar users by using button visuals and simplified inputs}
        \item {Provide users a customizable input notes list of their favourite characters, combos, matchups and maps}
        \item {Quick \& easy search and comparison between characters, their inputs, matchups and counters}
      \end{cvitems}
    }
    
%---------------------------------------------------------
  \cventry
    {\textbf{Flask, Javascript, MySQL}} % Job title
    {LoL Virtual Sandbox} % Organization
    {\href{https://github.com/basulaib/LoLSandbox}{\faGithub /LoLSandbox}} % Location
    {} % Date(s)
    {
      \begin{cvitems} % Description(s) of tasks/responsibilities
    %    \item {Stack: Django, Javascript, MySQL}
        \item {Online sandbox for items, runes, stats, champions to experiment with, combine different things and more}
        \item {Uses Riot Games API to pull data for users \& in-game information to accurately present up-to-date information}
        \item {Compare and contrast best-in-situation items and runes very quickly to provide an advantage}
        \item {Recommends the user the most popular builds and runes, and can export them as item sets in JSON format}
      \end{cvitems}
    }

%---------------------------------------------------------
  \cventry
    {\textbf{Python, Pandas, Numpy, Twitter API}} % Job title
    {Soccer Twitter Sentiment Analyzer} % Organization
    {\href{https://github.com/basulaib/SoccerTwitter-Sentiment}{\faGithub /SoccerTwitter-Sentiment}} % Location
    {} % Date(s)
    {
      \begin{cvitems} % Description(s) of tasks/responsibilities
    %    \item {Stack: Python, Pandas, Numpy, Twitter API}
        \item {Analyzes fan sentiment on Twitter for soccer teams during games to predict fan cheering moments, highlight moments, favorite players and controversial moments}
        \item {Lead and designed data flow and data format throughout the project to help the team work smoothly}
        \item {Handled real-time data streaming from Twitter API using Python with additional data cleaning to store as CSV sheets}
        \item {Collected data samples while watching soccer games to confirm the accuracy of the predictions}
        \item {Identified the limitations of the Twitter real-time API and communicated with the team to suggest solutions}
      \end{cvitems}
    }
    

%---------------------------------------------------------
  \cventry
    {\textbf{ANTLR4, Java, MIT Alloy Analyzer}} % Job title
    {ANTLR4 Compiler} % Organization
    {\href{https://github.com/basulaib/ANTLR4Compiler}{\faGithub /ANTLR4Compiler}} % Location
    {} % Date(s)
    {
      \begin{cvitems} % Description(s) of tasks/responsibilities
    %    \item {Stack: ANTLR4, Java, Alloy}
        \item {ANTLR4-based compiler that compiles user-defined regular language to predicate logic that Alloy uses to prove correctness}
        \item {Lead team \& communication by proposing plans, managing deadlines, assigning tasks and ensuring team members are motivated and comfortable}
        \item {Lead language design by proposing a simple and efficient regular language to simplify user interaction and processing work}
        \item {Designed test cases and inputs using Java to verify the output and detect any bugs or errors in the compiler}
      \end{cvitems}
    }
    

%---------------------------------------------------------
%  \cventry
%    {C} % Job title
%    {Unix Tree} % Organization
%    {\href{https://github.com/basulaib/tree}{\faGithub /tree}} % Location
%    {} % Date(s)
%    {
%      \begin{cvitems} % Description(s) of tasks/responsibilities
%    %    \item {Stack: C}
%        \item {Implementation of Unix command tree in C}
%      \end{cvitems}
%    }


%---------------------------------------------------------
\end{cventries}
